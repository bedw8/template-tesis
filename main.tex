% \documentclass{tesis}
\documentclass[spanish]{tesis}

% \usepackage[inkscape=onlynewer]{svg} % importar svg // la opción de inkscape es opcional. quitar si no se tiene inkscape

%%% carpeta de figuras y de svg's
\graphicspath{{figs}}
% \svgpath{{svg}}

%%%% recomendaciones

%% citas
\usepackage[noabbrev,nameinlink]{cleveref}
%% con este paquete podemos citar figuras, ecuaciones, tablas
%% sin tener que escribir "Figura \ref{arbol}", o "Ecuación \eqref{H}"
%%
%% en su lugar, el comando \Cref escribe automáticamente 
%% Figura, Ecuación, Tabla, identificando el tipo de las referencias 
%% ejemplo: \Cref{H} -> "Ecuación (1)"
%% el comando \cref hace lo mismo pero con la palabras en minúscula
%% (figura, ecuación, tabla), identificando el tipo de las referencias 

%% si el idioma del documento es ingles, el comando escribe las palabras en ingles. 

% descomentar para incluir bibliografía
%\usepackage[backend=biber,natbib=true,sorting=none,style=nature, maxcitenames=2,eprint=false,isbn=false,doi=true]{biblatex}
%\addbibresource{refs.bib}

\title{Titulo tesis}
\author{Juan Carlos Bodoque}
%\author[la]{Patana} se puede poner pronombres como argumento opcional 
\postgrado[Master/PhD/Magister/Doctorado]{mención en Locura y Lujuria}

%%% tutores/advisors - directores de Tesis
%% si hay más de un director pondrá "directores" (directors en ingles)
%% se puede modificar por directorAs o similares usando el argument opcional en una de las siguientes entradas
\anadirdireccion{Dr. AAAAAA }
\anadirdireccion{Dr. BBBBBB}
\anadirdireccion[as]{Dr. CCCCC}

%%% miembros comision evaluadora
\anadircomision{Dr. Nombre1 Ap}
\anadircomision{Dr. Nombre2 Ap}

%%% cambiar por fecha de aprobacion
\aprobacionfecha{YYYY-MM-DD}


%%%%% si se quiere compilar solo una parte del documento, es cosa de indicarlo con
% \includeonly{./intro.tex}

%%%%%%%%%%%%%%%%%%%%%%%%%%%%%%%%%%%%%%%%%%%%%%%%%%
%%%%%%%%%%%%%%%%%%%%%%%%%%%%%%%%%%%%%%%%%%%%%%%%%%
\begin{document}

%%% estructura en distintos archivos
% comandos para crear portada y pagina de aprobacion
\maketapa 
\makeaprueba
 % incluye portada y pagina de aprobacion
\null
\vfill
\begin{flushright}
% Dedicatoria opcional
 \emph{Optional dedication.}
\end{flushright}

\cleardoublepage

\psection{Biography/Biografía}
\lipsum[1-1]
 % hay que incluir biografia. incluye pagina de dedicatoria
% agradecimientos
\psection{Acknoledgements/Agradecimientos}
\lipsum[2]

% en caso de escribir tesis en ingles, igual poner un resumen en español
\newpage
\psection{Abstract}
\lipsum[1]

\psection{Resumen}
\lipsum[1]


    


\cleardoublepage % para que quede en hoja par
% en caso de escribir tesis en ingles, igual poner un resumen en español
\psection{Abstract}
\lipsum[1]

\psection{Resumen}
\lipsum[1]


%%% indice en hoja impar (lado izquierdo)
% \cleardoublepage % para que haya una hoja blanca antes del indice
% \begingroup
% \let\cleardoublepage\clearevenpage
% \tableofcontents
% \endgroup

%%% indice en hoja par (lado derecho)
\clearevenpage % para que haya una hoja blanca antes del indice
\tableofcontents

%%% CONTENIDO %%%
% \clearevenpage % para que haya una hoja blanca después del indice
\cleardoublepage % poner esto al inicio de cada capitulo si
% se quiere forzar que parta en una pagina par (lado derecho)
\pagestyle{everypage}
%%% aquí ya se pueden crear los archivos que uno quiera
\import{./}{intro}
\import{./}{analisis}
\import{./}{discusion}

%% descomentar para mostrar referencias
% \cleardoublepage
% \printbibliography[heading=bibintoc] 
% \cleardoublepage

%% apéndice
% \appendix
% \chapter{Additional figures}

\section{AA}
\lipsum[2]

\section{BB}
\lipsum[2]

\section{CC}
\lipsum[2]



% pagina blanca adicional
\thispagestyle{empty}
\clearevenpage
\end{document}

